% Document definition
\documentclass[a4paper,11pt,openany]{book}

% Packages
	% General packages
	\usepackage[utf8x]{inputenc}								% Danish language support
	\usepackage{sectsty}											% Makes it possible to manipulate fonts
	\usepackage[light]{antpolt}									% Provides awesome font
	\usepackage{hyperref}											% Provides \url and clickable links in the pdf
	\usepackage{cleveref}											% Provides \cref and \Cref for including text before references
	\usepackage{graphicx}										% Provides the \includegraphics[]{} command
	\usepackage{comment}										% Provides comment-environment for multi-line commenting
	\usepackage{twoopt}											% Allows adding commands with two optional arguments
	\usepackage{natbib}											% Provides \citep for bibliography referencing
	\usepackage{setspace}											% Provides onehalfspacing environment
	\usepackage{enumitem}										% Provides control of spacing in lists
	\usepackage{amsbsy}											% Provides \boldsymbol command
	\usepackage{lastpage}											% Provides LastPage variable
	
	% Math packages
	\usepackage{amsmath, amssymb, amsfonts}		% Math symbolic jargon
	\usepackage{amsthm}											% Theorem environment
	\usepackage{mathrsfs}										% Provides the \mathscr{} curly font
	\usepackage{stmaryrd}										% Provides the \lightning symbol, among others
	\usepackage{proof}												% Natural deduction with \infer
	\usepackage{tikz}												% Awesome diagrams
		\usetikzlibrary{matrix,arrows}								% Matrices and arrows for style points
		\tikzset{middlearrow/.style={								% Middle arrows for quotient spaces
        	decoration={markings, mark=at
        		position 0.5 with {\arrow{#1}}},
        	postaction={decorate}
    	}}
		
	% Page layout
	\usepackage{fullpage}											% Reduces margins
	\usepackage{wallpaper}										% Adds the \ThisLRCornerWallPaper command
	\usepackage{fancyhdr}										% Provides headers
	\usepackage[T1]{fontenc}									% Used for vertical line in chapter headings
	\usepackage{titlesec, blindtext, color}					% Chapter heading setup
	\usepackage{changepage}									% Provides adjustwidth environment for claims

% Font size
\sectionfont{\large}
\chapterfont{\LARGE}

% Titlesec setup
\definecolor{gray75}{gray}{0.75}
\titleformat{\chapter}[hang]{\huge\scshape}{\thechapter\hspace{20pt}\textcolor{gray75}{|}\hspace{20pt}}{0pt}{\huge\scshape}
\titlespacing*{\chapter}{0pt}{0pt}{40pt}
\titleformat{\section}[hang]{\Large\scshape}{\thesection}{2ex}{}[]

% Prevent widows and orphans
\widowpenalty = 10000
\clubpenalty = 10000

% List spacing
\setlist{nolistsep}

% No indent
\setlength{\parindent}{0in}

% Theorem environment
\newtheoremstyle{scthmstyle}		% Name
	{15pt}										% Space above
	{15pt}										% Space below
	{\itshape}									% Body font
	{}												% Indent amount
	{\scshape\bfseries}					% Theorem head font
	{.}												% Punctuation after theorem head
	{.5em}										% Space after theorem head
	{}												% Theorem head spec (can be left empty, meaning ënormalí)

\newtheoremstyle{scdefstyle} 		% Name
	{20pt}										% Space above
	{20pt}										% Space below
	{\normalfont}							% Body font
	{}												% Indent amount
	{\scshape\bfseries}					% Theorem head font
	{.}												% Punctuation after theorem head
	{.5em}										% Space after theorem head
	{}												% Theorem head spec (can be left empty, meaning ënormalí)

\newtheoremstyle{scremstyle} 	% Name
	{15pt}										% Space above
	{15pt}										% Space below
	{\normalfont}							% Body font
	{}												% Indent amount
	{\scshape\bfseries}					% Theorem head font
	{.}												% Punctuation after theorem head
	{.5em}										% Space after theorem head
	{}												% Theorem head spec (can be left empty, meaning ënormalí)

\newtheoremstyle{scclaistyle} 		% Name
	{15pt}										% Space above
	{15pt}										% Space below
	{\normalfont}							% Body font
	{0.5cm}										% Indent amount
	{\scshape\bfseries}					% Theorem head font
	{.}												% Punctuation after theorem head
	{.5em}										% Space after theorem head
	{}												% Theorem head spec (can be left empty, meaning ënormalí)
	
\theoremstyle{scthmstyle}
\newtheorem{theorem}[subsection]{Theorem}
\newtheorem{proposition}[subsection]{Proposition}
\newtheorem{lemma}[subsection]{Lemma}
\newtheorem{corollary}[subsection]{Corollary}
\newtheorem{conjecture}[subsection]{Conjecture}
\theoremstyle{scdefstyle}
\newtheorem{definition}[subsection]{Definition}
\newtheorem{example}[subsection]{Example}
\theoremstyle{scremstyle}
\newtheorem{remark}[subsection]{Remark}
\theoremstyle{scclaistyle}
\newtheorem{claim}[subsubsection]{Claim}

% User-defined commands
	% General things	
	\newcommand{\eq}[1]{\begin{align*} #1 \end{align*}}
	\newcommand{\eqq}[1]{\begin{align*} #1\\ \end{align*}}
	\newcommand{\pic}[2]{\begin{center}\includegraphics[scale=#2]{#1.jpg}\\\end{center}}
	\renewcommand{\labelenumi}{(\roman{enumi}) }	% Using roman numerals in lists
	\renewcommand{\b}[1]{{\bf #1}}
	\newcommand{\abstract}[1]{\begin{quote}{\footnotesize\textsc{Abstract.} #1}\\\end{quote}}
	
	% Theorem environments
	\newcommandtwoopt{\theo}[3][][]{\begin{theorem}[#1]\label[theorem]{#2}#3\end{theorem}}
	\newcommandtwoopt{\prop}[3][][]{\begin{proposition}[#1]\label[proposition]{#2}#3\end{proposition}}
	\newcommandtwoopt{\lemm}[3][][]{\begin{lemma}[#1]\label[lemma]{#2}#3\end{lemma}}
	\newcommandtwoopt{\coro}[3][][]{\begin{corollary}[#1]\label[corollary]{#2}#3\end{corollary}}
	\newcommandtwoopt{\conj}[3][][]{\begin{conjecture}[#1]\label[conjecture]{#2}#3\end{conjecture}}
	\newcommandtwoopt{\defi}[3][][]{\begin{definition}[#1]\label[definition]{#2}#3\end{definition}}
	\newcommandtwoopt{\exam}[3][][]{\begin{example}[#1]\label[example]{#2}#3\end{example}}
	\newcommandtwoopt{\rema}[3][][]{\begin{remark}[#1]\label[remark]{#2}#3\end{remark}}
	
	\newcommandtwoopt{\qtheo}[3][][]{\begin{theorem}[#1]\label[theorem]{#2}#3$\qed$\end{theorem}}
	\newcommandtwoopt{\qprop}[3][][]{\begin{proposition}[#1]\label[proposition]{#2}#3$\qed$\end{proposition}}
	\newcommandtwoopt{\qlemm}[3][][]{\begin{lemma}[#1]\label[lemma]{#2}#3$\qed$\end{lemma}}
	\newcommandtwoopt{\qcoro}[3][][]{\begin{corollary}[#1]\label[corollary]{#2}#3$\qed$\end{corollary}}
	
	\newcommand{\proofretard}{\textsc{Proof.} }
	\renewcommand{\proof}[1]{\textsc{Proof.} #1$\qed$\\}
	\newcommand{\clai}[3][]{\begin{claim}[#1]#2\end{claim}}
	\newcommand{\cproof}[1]{\begin{adjustwidth}{0.5cm}{0pt}\textsc{Proof of claim.} #1$\hfill\diamondsuit$\\\end{adjustwidth}}
	\renewcommand{\qed}{\hfill\blacksquare}
	\newcommand{\qedeq}{\tag*{$\blacksquare$}}
	
	% Declared operators
	\DeclareMathOperator{\sgn}{sgn}
	\DeclareMathOperator{\lcm}{lcm}
	\DeclareMathOperator{\ran}{ran}
	\DeclareMathOperator{\cod}{cod}
	\DeclareMathOperator{\dom}{dom}	
	\DeclareMathOperator{\cond}{cond}
	\DeclareMathOperator{\rank}{rank}
	\DeclareMathOperator{\xor}{\oplus}
	\DeclareMathOperator{\nor}{\downarrow}
	\DeclareMathOperator{\nand}{\uparrow}
	\DeclareMathOperator{\biglor}{\bigvee}
	\DeclareMathOperator{\bigland}{\bigwedge}
	\DeclareMathOperator{\Lr}{\Leftrightarrow}
	\DeclareMathOperator{\lr}{\leftrightarrow}
	\DeclareMathOperator{\ip}{\perp\!\!\!\perp}
	\DeclareMathOperator{\psubset}{\subsetneq}
	\DeclareMathOperator{\psupset}{\supsetneq}
	\DeclareMathOperator{\elsub}{\preceq}
	\DeclareMathOperator{\elsup}{\succeq}
	\DeclareMathOperator{\pelsub}{\prec}
	\DeclareMathOperator{\pelsup}{\succ}
	\DeclareMathOperator{\nmodels}{\nvDash}
	\DeclareMathOperator{\contr}{\lightning}
	\DeclareMathOperator{\nsubset}{\nsubseteq}
	\DeclareMathOperator{\nsupset}{\nsupseteq}
	\DeclareMathOperator{\proves}{\vdash}
	\DeclareMathOperator{\nproves}{\nvdash}
	\DeclareMathOperator{\biproves}{\dashv\vdash}
	\DeclareMathOperator{\gal}{Gal}
	\DeclareMathOperator{\ac}{AC}
	\DeclareMathOperator{\zf}{ZF}
	\DeclareMathOperator{\zfc}{ZFC}
	\DeclareMathOperator{\gch}{GCH}
	\DeclareMathOperator{\ch}{CH}
	\DeclareMathOperator{\on}{On}
	\DeclareMathOperator{\con}{Con}
	\DeclareMathOperator{\cf}{cf}
	\DeclareMathOperator{\im}{Im}
	\DeclareMathOperator{\sh}{SH}
	\DeclareMathOperator{\kh}{KH}
	\DeclareMathOperator{\restr}{\upharpoonright}
	\DeclareMathOperator{\Def}{Def}
	\DeclareMathOperator{\id}{id}
	\DeclareMathOperator{\ot}{ot}
	\DeclareMathOperator{\fst}{fst}
	\DeclareMathOperator{\snd}{snd}
	\DeclareMathOperator{\fin}{Fin}
	\DeclareMathOperator{\ds}{DS}
	\DeclareMathOperator{\Bound}{Bound}
	\DeclareMathOperator{\height}{ht}
	\DeclareMathOperator{\ult}{Ult}
	\DeclareMathOperator{\crit}{crit}
	\DeclareMathOperator{\bSigma}{\boldsymbol\Sigma}
	\DeclareMathOperator{\bPi}{\boldsymbol\Pi}
	\DeclareMathOperator{\bDelta}{\boldsymbol\Delta}
	
	% Redeclared operators
	\renewcommand{\subset}{\subseteq}
	\renewcommand{\supset}{\supseteq}
	\renewcommand{\char}{\text{char}}
	
	% Convenient shortcuts
	\newcommand{\vto}[2]{\begin{pmatrix}#1\\#2\end{pmatrix}}
	\newcommand{\vtre}[3]{\begin{pmatrix}#1\\#2\\#3\end{pmatrix}}
	\newcommand{\mto}[4]{\begin{pmatrix} #1 & #2 \\ #3 & #4\end{pmatrix}}
	\newcommand{\mtre}[9]{\begin{pmatrix} #1 & #2 & #3 \\ #4 & #5 & #6 \\ #7 & #8 & #9\end{pmatrix}}
	\newcommand{\bra}[1]{\langle #1\rangle}
	\newcommand{\norm}[1]{\left|\left|#1\right|\right|}
	\newcommand{\godel}[1]{\ulcorner #1 \urcorner}
	\newcommand{\down}[1]{{\downarrow #1}}
	\newcommand{\up}[1]{{\uparrow #1}}
	\newcommand{\los}{{\fontfamily{arial}\selectfont \L}o\' s}
