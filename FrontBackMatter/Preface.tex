\vspace*{0pt}

\begin{adjustwidth}{7cm}{0cm}
\begin{quote}
``\textit{A favorite example against the pragmatic view that we accept an axiom because of its elegance (simplicity) and power (usefulness) is the constructibility hypothesis. It should be accepted according to the pragmatic view but is not generally accepted as true.}"\\

 - Hao Wang \citep{Wang}\\\\
\end{quote}
\end{adjustwidth}

\section*{\huge Preface}
\thispagestyle{fancy}
\setlength{\parindent}{18pt}

\begin{onehalfspacing}

This text is intended as an introduction to constructibility theory, and it is assumed that readers have basic knowledge of set theory at the level of ordinals and cardinals, as well as knowledge of model theory up to the Löwenheim-Skolem Theorem; these facts will only be briefly recalled, mostly without proof. I do not assume prior knowledge about large cardinals, tree theory, ultraproducts or proof theory, and will explain what is needed from these fields slightly more detailed.

My notation used is pretty standard. I denote tuples in angled brackets $\bra{x_1,\hdots,x_n}$, languages in calligraphic font $\mathcal{L}$ and models in fraktur font $\mathfrak{A}=\bra{A,\hdots}$. If two structures $\mathfrak{A},\mathfrak{B}$ are isomorphic, I denote this by $\mathfrak{A}\cong\mathfrak{B}$, and I write $x\approx y$ to indicate that the sets $x$ and $y$ are equinumerous (i.e. that there exists a bijection between them). Logical equivalence is denoted $\varphi\equiv\psi$. I use both $\omega_n$ and $\aleph_n$ to denote the $n$'th infinite cardinal, where $\omega_n$ will be used to emphasize ordinal properties and $\aleph_n$ cardinal properties. The identity class function is denoted $\id:=\{\bra{x,y}\mid x=y\}$; the identity function on a set $x$ is thus denoted $\id\restr x$. $\dom f$ and $\ran f$ denotes the domain and range of a function $f$, respectively, and I use the abbreviation $f"x:=\ran(f\restr x)$. $\on$ denotes the class of ordinals and I'll write $``\alpha\in\on"$ as an abbreviation for ``$\alpha$ is an ordinal". I will use the notation $\vec{v}$ for $v_1,\hdots,v_n$ (so it's not an $n$-tuple). I work in first-order logic with the adequate set $\{\land,\lnot,\exists\}$ of logical symbols. $\zfc$ is assumed throughout, and if a theorem $\sigma$ is proved by assuming a different set of axioms $T$, I will write this as $T\proves\sigma$. I write $\contr$ to denote a contradiction.

I would like to sincerely thank Asger Törnquist for his help and guidance, as well as David Schrittesser, Martin Speirs and Nicolas Bru Frantzen for carefully reading through the thesis and giving a lot of constructive criticism.

\end{onehalfspacing}

\setlength{\parindent}{0pt}